\section{Schlusswort}
Im Verlaufe dieser Arbeit hat sich gezeigt, dass die Realisierung eines qualitativ hochwertigen Baluns viel Erfahrung benötigt. Die Dimensionierung basiert auf vielen Abschätzungen und Annahmen. Da es keine allgemein gültige Anleitung zur Dimensionierung eines Baluns gibt, gestattet sich die Definition eines geeigneten Models als heikel. Einige Effekte, welche man im Model vernachlässigt hat, haben trotzdem einen Einfluss und müssen später durch Nachjustierung korrigiert werden. Dies gilt besonders bei der Verwendung von gekoppelten Spulen und deren Herstellung von Hand. Gerade das Wickeln der Spulen hat gezeigt, welchen Einfluss auf die späteren Eigenschaften der Anpressdruck und Abstand jeder einzelnen Windung hat.\newline

Gerade die Tatsache, dass es kein \glqq Kochrezept\grqq für Baluns gibt, machte die Arbeit anspruchsvoll und wir konnten viel Gelerntes einfliessen lassen.
Schliesslich war es für uns eine spannende Thematik und interessante Arbeit von der Dimensionierung bis zur Realisierung und Validierung.