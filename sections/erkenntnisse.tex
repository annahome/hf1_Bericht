\section{Erkenntnisse}
Die Validierung hat folgendes aufgezeigt: Bei der gewählten Arbeitsfrequenz von \SI{20}{MHz} erzwingt der Guanella-Balun wie gewünscht symmetrische Ströme. Die Impedanzanpassung funktioniert jedoch nur bis zu einer Frequenz von ca. \SI{10}{MHz}. Diese obere Grenzfrequenz wird durch das Verhältnis von Wicklungskapazität zu Induktivität bestimmt, und kann deshalb wie folgt erhöht werden:
\begin{itemize}
	\item Abstand zwischen der Wicklungen erhöhen
	\item Anzahl der Wicklungen erhöhen
\end{itemize}

Damit solche Änderungen keine Einflüsse auf die Resonanzfrequenz des Baluns haben, muss ein anderer Parameter, wie zum Beispiel der Kern, entsprechend verändert werden.\\
\vspace{1cm}
Möchte man den realisierten Balun ohne Änderungen verwenden, muss eine andere Arbeitsfrequenz gewählt werden. In Frage kommen die Frequenzbereiche, bei welchen sowohl die Impedanzanpassung als auch die Symmetriebedingung erfüllt ist. Dies ist ungefähr zwischen \SI{5}{MHz} und \SI{10}{MHz} der Fall. 
\vspace{1cm}

